\documentclass[12pt,twoside]{report}

% some definitions for the title page
\newcommand{\reporttitle}{Transfer Learning for Deep Learning Radiotherapy Planning}
\newcommand{\reportauthor}{Anton Zhitomirsky}
\newcommand{\supervisor}{Prof Ben Glocker}
\newcommand{\secondMarker}{Dr Thomas Heinis}
\newcommand{\reporttype}{MEng Individual Project}

% load some definitions and default packages
%%%%%%%%%%%%%%%%%%%%%%%%%%%%%%%%%%%%%%%%%
% University Assignment Title Page 
% LaTeX Template
% Version 1.0 (27/12/12)
%
% This template has been downloaded from:
% http://www.LaTeXTemplates.com
%
% Original author:
% WikiBooks (http://en.wikibooks.org/wiki/LaTeX/Title_Creation)
%
% License:
% CC BY-NC-SA 3.0 (http://creativecommons.org/licenses/by-nc-sa/3.0/)
% 
%
%%%%%%%%%%%%%%%%%%%%%%%%%%%%%%%%%%%%%%%%%
%----------------------------------------------------------------------------------------
%	PACKAGES AND OTHER DOCUMENT CONFIGURATIONS
%----------------------------------------------------------------------------------------
\usepackage[a4paper,hmargin=2.0cm,vmargin=1.0cm,includeheadfoot]{geometry}
\usepackage{textpos}

\usepackage[square,numbers]{natbib} % for bibliography
\usepackage[nottoc]{tocbibind} % Includes "References" in the table of contents
\bibliographystyle{unsrtnat}

\usepackage{tabularx,longtable,multirow,subfigure,caption}%hangcaption
\usepackage{makecell}
\usepackage{fancyhdr} % page layout
\usepackage{url} % URLs
\usepackage[english]{babel}
\usepackage{amsmath}
\usepackage{graphicx}
\usepackage{adjustbox}
\usepackage{scalerel}
\usepackage{pdflscape}
\usepackage{dsfont}
\usepackage{epstopdf} % automatically replace .eps with .pdf in graphics
\usepackage{backref} % needed for citations
\usepackage{array}
\usepackage{latexsym}

\usepackage[pdftex,pagebackref,hypertexnames=false,colorlinks]{hyperref} % provide links in pdf
\usepackage{booktabs}
\usepackage{wrapfig}
\usepackage{caption}  % Required for \captionof
\usepackage{float} % for H option in figures
\usepackage{amssymb}
\usepackage{amsmath}
\usepackage{csquotes}
% \usepackage{subcaption} % causes a compilation error after changing back to natbib referencing... 

\hypersetup{pdftitle={},
  pdfsubject={}, 
  pdfauthor={},
  pdfkeywords={}, 
  pdfstartview=FitH,
  pdfpagemode={UseOutlines},% None, FullScreen, UseOutlines
  bookmarksnumbered=true, bookmarksopen=true, colorlinks,
    citecolor=black,%
    filecolor=black,%
    linkcolor=black,%
    urlcolor=black}

\usepackage[all]{hypcap}

%%%%%%%%%%%%%%%%%%%%%%%%%%%%%%%%%%%%%%%%%%%%%%%%%%%%%%%%%%%%%%%%%%%%%%%%%%%%%%%%
% LISTINGS ammendments
%%%%%%%%%%%%%%%%%%%%%%%%%%%%%%%%%%%%%%%%%%%%%%%%%%%%%%%%%%%%%%%%%%%%%%%%%%%%%%%%
\usepackage{listings}

\lstset
{ %Formatting for code in appendix
    language=Matlab,
    basicstyle=\footnotesize,
    % numbers=left,
    stepnumber=1,
    showstringspaces=false,
    tabsize=1,
    breaklines=true,
    breakatwhitespace=false,
    frame=single,
    columns=fullflexible,
    postbreak=\mbox{\textcolor{red}{$\hookrightarrow$}\space},
}

%\usepackage{color}
%\usepackage[tight,ugly]{units}
%\usepackage{float}
%\usepackage{tcolorbox}
%\usepackage[colorinlistoftodos]{todonotes}
% \usepackage{ntheorem}
% \theoremstyle{break}
% \newtheorem{lemma}{Lemma}
% \newtheorem{theorem}{Theorem}
% \newtheorem{remark}{Remark}
% \newtheorem{definition}{Definition}
% \newtheorem{proof}{Proof}


%%% Default fonts
\renewcommand*{\rmdefault}{bch}
\renewcommand*{\ttdefault}{cmtt}


%%% Default settings (page layout)
\setlength{\parindent}{0em}  % indentation of paragraph
\setlength{\parskip}{.3em}

% \setlength{\parindent}{0em}  % indentation of paragraph

\setlength{\headheight}{14.5pt}
\pagestyle{fancy}
\renewcommand{\chaptermark}[1]{\markboth{\chaptername\ \thechapter.\ #1}{}}
%\fancyhead[RO]{\sffamily \textbf{\thepage}} %Page no.in the right on even pages
%\fancyhead[LE]{\sffamily \textbf{\thepage}} %Page no. in the left on odd pages

\fancyfoot[ER,OL]{\thepage}%Page no. in the left on
%odd pages and on right on even pages
\fancyfoot[OC,EC]{\sffamily }
\renewcommand{\headrulewidth}{0.1pt}
\renewcommand{\footrulewidth}{0.1pt}
\captionsetup{margin=10pt,font=small,labelfont=bf}


%--- chapter heading

\def\@makechapterhead#1{%
  \vspace*{10\p@}%
  {\parindent \z@ \raggedright \sffamily
    \interlinepenalty\@M
    \Huge\bfseries \thechapter \space\space #1\par\nobreak
    \vskip 30\p@
  }}

%--- chapter heading

\def\@makechapterhead#1{%
  \vspace*{10\p@}%
  {\parindent \z@ \raggedright \sffamily
    %{\Large \MakeUppercase{\@chapapp} \space \thechapter}
    %\\
    %\hrulefill
    %\par\nobreak
    %\vskip 10\p@
    \interlinepenalty\@M
    \Huge\bfseries \thechapter \space\space #1\par\nobreak
    \vskip 30\p@
  }}

%---chapter heading for \chapter*  
\def\@makeschapterhead#1{%
  \vspace*{10\p@}%
  {\parindent \z@ \raggedright
    \sffamily
    \interlinepenalty\@M
    \Huge \bfseries  #1\par\nobreak
    \vskip 30\p@
  }}
\allowdisplaybreaks

% load some macros
\input{../.latex-templates/notation}

% load title page
\begin{document}
\input{../.latex-templates/titlepage}

% page numbering etc.
\pagenumbering{roman}
\clearpage{\pagestyle{empty}\cleardoublepage}
\setcounter{page}{1}
\pagestyle{fancy}

% \cleardoublepage
%%%%%%%%%%%%%%%%%%%%%%%%%%%%%%%%%%%%
% \section*{Acknowledgments}
% Comment this out if not needed.

% \clearpage{\pagestyle{empty}\cleardoublepage}

%%%%%%%%%%%%%%%%%%%%%%%%%%%%%%%%%%%%

\begin{abstract}\label{sect:abstract}
  Radiotherapy planning involves outlining the macro and microscopic spreads of a cancer. Previously, bespoke models were trained on sufficient sample sizes of architectures that vary significantly in design to address distinct segmentation objectives. However, this makes models inaccessible to departments with limited datasets, increasing the barrier to experiencing the promised superiority beyond a model's specific use case.

  We, therefore, analyse the effectiveness of transfer learning techniques in leveraging knowledge learned from one domain into another. Variants of transfer include zero-shot, few-shot and many-shot transfer learning. The evaluation of n-shot approaches of transfer learning onto the radiotherapy planning objective will clarify whether the transfer is a valid strategy for increasing performance over models learnt from scratch.

  We represent three distinct model architectures for each type of transfer. We conclude that a many-shot transfer using TotalSegmentator [TODO]. We find that few-shot transfer using UinverSeg does not apply to the complexity of 3-D volume delineations for cervical cancer. Finally, the finetuning of zero-shot models, represented by MedSAM, offers local bounded improvements of volume delineation independently of tumour localisation.

  % The repo is available at \url{https://github.com/AntonZhitomirsky/radiotherapy}.
\end{abstract}
  
%%%%%%%%%%%%%%%%%%%%%%%%%%%%%%%%%%%%
%--- table of contents
\fancyhead[RE,LO]{\sffamily {Table of Contents}}
\tableofcontents
  
% \clearpage{\pagestyle{empty}\cleardoublepage}

%%%%%%%%%%%%%%%%%%%%%%%%%%%%%%%%%%%%

\pagenumbering{arabic}
\setcounter{page}{1}
\fancyhead[LE,RO]{\slshape \rightmark}
\fancyhead[LO,RE]{\slshape \leftmark}

%%%%%%%%%%%%%%%%%%%%%%%%%%%%%%%%%%%%
\chapter{Introduction}

\section{Technical Context}

Transfer Learning is a technique in which a model trained on one task is repurposed and fine-tuned on a second task. Thus, the model can leverage the knowledge gained from the first task to improve performance on the second task. An accrual of literature across PubMed in 2023-2024 was conducted with a search string displayed in Figure~\ref{fig:pubmed-search-string}, which found a gap in technical research surrounding Transfer Learning in the radiotherapy planning domain. 

\begin{figure}[H]
  \begin{lstlisting}
    ``radiotherapy'' AND ``contour'' AND (``cervix'' OR ``cervical'') AND ``cancer'' AND (``Deep Learning'' OR ``DeepLearning'' OR ``Machine Learning'' OR ``ML'' OR ``Artificial Intelligence'' OR ``AI'' OR ``Computer assisted'')  
   \end{lstlisting}
   \caption{Search string used in PubMed accrual in 2023-2024}\label{fig:pubmed-search-string}  
\end{figure}

The conclusion of the accrual confirmed that the most common network structure for radiotherapy planning was the U-Net architecture~\cite{Samarasinghe2021-ps,Lin2021-oz,Sartor2020-et,LIU2020184,Rhee2020-ms,LIU2020172}. However, each segmentation benchmark required modifications such as specialized architectures or bespoke training schemes to achieve competitive performance. These tailored modifications contribute little to the overall development in medical segmentation~\cite{isensee2024nnunet}, as it becomes increasingly challenging for researchers to identify methods that live up to their promised superiority beyond the limited scenarios they are demonstrated on~\cite{nnunet}.

The publications from the accrual also reveal that the bespoke trained model assumes a sufficient sample size, representing the entire population of all individuals under the same circumstances. However, the underrepresented users of these models cannot apply this assumption; most hospital teams work in a specialized medical field with only in-house patients who follow specific hospital guidelines. A simple plug-and-play approach is unsuitable in this scenario because segmentation guidelines often differ between facilities. Therefore, specific recalibration is necessary to effectively use these models because incorrect or inaccurate contours are the primary factors contributing to treatment failures in radiotherapy~\cite{Rhee2020-ms}.

We therefore analyse the effectiveness of Transfer Learning in radiotherapy planning. The Transfer Learning approach aims to leverage other trained models with great performance and lift low-level features for the target domain. A PubMed search for publications mentioning transfer learning in the radiotherapy planning field returned no relevant matches. Therefore, research on transfer in the medical domain will also fill the literature gap, which has yet to consider solving the auto-contouring problem for radiotherapy planning volumes. 

\section{Objective and Methodology}

Firstly, we select a dataset from The Royal Marsden Hospital, which will act as the pillar for advocating the success and use of transfer learning in radiotherapy planning. The real-world clinical dataset provides segmentations for key anatomies and tumours that aid in radiotherapy planning for females with cervical cancer. It has yet to gain exposure to the widespread segmentation challenges and has uncommon and limited segmentation patterns specified by internal team-wide guidelines.

Secondly, we analyse the effectiveness of Transfer learning in radiotherapy planning. To research the application of this technique in the medical context, we study three types of modern training approaches to study their transferability. These types are zero-shot, few-shot, and many-shot learning transfers.

\section{Results and Conclusions}

\section{Outline of Report}

The report will first discuss the relevant background knowledge required for this project in Chapter~\ref{sect:motivation}. Within, we provide a high-level overview of anatomies and the clinical context (Section~\ref{sect:clinical-context}) as well as core existing academic knowledge in the Computer-assisted vision in the Medical Imaging field (Section~\ref{sect:machine-learning-for-image-segmentation}-\ref{sect:performance-evaluation}). 

Then, Chapter~\ref{sect:methodology} describes the experiments used to evaluate how well different architectures transfer knowledge into the target domain, as well as an overview of the results (Chapter~\ref{sect:results}) and a discussion of their implications (Chapter~\ref{sect:discussion}).

%%%%%%%%%%%%%%%%%%%%%%%%%%%%%%%%%%%%
\chapter{Motivation}\label{sect:motivation}

\section{Clinical Context}\label{sect:clinical-context}

This project will have its foundation for experimentation in a dataset provided by the Royal Marsden Hospital~\cite{AMLART-data}. The real-world clinical dataset segments key anatomies and tumours that aid in radiotherapy planning for females with cervical cancer. It has yet to gain exposure to the widespread segmentation challenges and has uncommon and limited segmentation patterns. This dataset will act as the pillar for justifying the success of the transferability of knowledge between medical domains.

In this section, we discuss the clinical context behind cervical cancer in the population, the Hospital's pipeline for segmenting patients in preparation for radiotherapy treatment, and the Hospital's motivation for recruiting an AI tool to assist in its treatment pipeline. 

\subsection{Cervical Cancer}\label{sect:cervical-cancer}

Cancer is a burden around the globe that has been a driver for almost one-sixth of the world's mortality in 2022~\cite{Global-cancer-2022}. In females, cervical cancer makes up 25 countries' leading causes of cancer death, following breast cancer for 157 countries in 2022~\cite{Global-cancer-2022}. Furthermore, an estimated 1 million maternal orphans who lose their mothers to cancer suffer long-term disadvantages in health and education~\cite{Guida2022}. Thankfully, cancer screening services provided by hospitals around Europe have been shown to decrease incidence and mortality rates of cervical cancer in women over the recent years~\cite{Global-cancer-2022}, which inspires further complete clinical understanding of the disease. Paired with quality improvements offered by medical imaging models, this forms the motivation for total control over cervical cancer.

\subsection{Radiotherapy Treatment}

A mechanism available for cancer treatment involves radiation therapy. High beams of radiation energy are tuned to hone in to target cancerous cells in a clinically defined `target area'. The cells killed by the energy experience interphase or proliferative death depending on the cell cycle stage. Death occurs when the damage to genetic material within the cell prevents it from dividing, or the cell's accumulation of genetic aberrations leads to a ``mitotic catastrophe''~\cite{cell-death}. In Europe, radiotherapy treatment was used on average for 70\% of cases, with a curative rate of 40\%~\cite{radiotherapy-advances, Thompson2018}.

This death is characteristic of any cell subjected to high energy beams, placing much responsibility on the Oncologist to deliver an accurate treatment area so that healthy cells are unaffected; any adverse alteration of an organ's standard functionality may cause grave implications for the already compromised patient. Therefore, the precise and complex nature of the task is estimated to take oncologists 90--120 mins to delineate target areas for radiotherapy~\cite{LIU2020184}.

This time-consuming endeavour is never favourable for a patient already in a dangerous situation. For mid-low-income countries, where this may not be an available resource, this leaves them with a death rate 18 times that of a higher-income country~\cite{cervical-cancer-epidemic}. 

\subsection{CT modality}\label{sect:ct-modality}

High-resolution and high-contrast CT machines have further benefited cancer treatment due to their noninvasive nature and ability to view patients' internal organs. X-ray devices rotate around a specified body part, and computer-generated cross-sectional images are produced~\cite{file-formats}. Whilst the scanner rotates, the patient's table slowly moves up and down inside the tube to produce different cross-section images. The images show damaged and surrounding soft tissue, allowing physicians to propose clinical target volumes more accurately.

\subsubsection{Hounsfield Units}

\begin{figure}[H]
    \centering
    \subfigure[Muscle Window $(35,55)$~\cite{other-HD-units}]{{\includegraphics[width=0.3\textwidth, trim=0.2cm 1cm 21cm 2cm, clip]{../figures/HU-window.png}}}
    \subfigure[Bone Window  $(300,400)$~\cite{cancellous-bone}]{{\includegraphics[width=0.3\textwidth, trim=10.6cm 1cm 10.6cm 2cm, clip]{../figures/HU-window.png}}}
    \subfigure[Fat Window $(-120, -90)$~\cite{other-HD-units}]{{\includegraphics[width=0.3\textwidth, trim=21cm 1cm 0.2cm 2cm, clip]{../figures/HU-window.png}}}
    \caption{Coronal view the same image slice of a CT image, with different window cropping (Patient id: 49, slice 251). White areas represent high-density tissues, and black areas represent low-density tissues within the window range.}\label{fig:ct-windows}
\end{figure}

The operator or physician decides the granularity or image slice thickness, which ranges from 1mm to 10mm. Therefore, the precision along each axis creates a cube, or voxel, representing the value on a grid in three-dimensional space. The voxel values are measured in Hounsfield Units (HU)~\cite{diagnostic-radiology-physics}. 

Contrary to natural images, where pixel values vary from 0 to 255 in 3 channels representing Red, Blue and Green, the Hounsfield scale is a quantitative scale describing radiodensity. The image intensity reflects tissue type; each voxel intensity refers to a specific tissue composition. The positive values (white) result from more dense tissue with greater X-ray beam absorption, and negative values (black) are less dense tissue with less X-ray beam absorption~\cite{Statpearls}.  

Therefore, because the HU scale is relative, different windows may be taken for a CT scan to highlight different tissues. Those voxels within the window will likely be tissues of a specific classification. For example, as shown in Figure~\ref{fig:ct-windows}, we display three such windows: muscle, cancellous bone and fat.

\subsection{Radiotherapy Planning}\label{sect:radiotherapy-planning}

\begin{figure}[H]
  \subfigure[GTV]{
    \includegraphics[width=.23\linewidth, trim=35 905 562 40, clip]{../figures/planning-areas.jpeg}\label{fig:gtv}
  }
  \subfigure[GTV macroscopic]{
    \includegraphics[width=.23\linewidth, trim=562 905 35 40, clip]{../figures/planning-areas.jpeg}\label{fig:gtv-macroscopic}
  }
  \subfigure[CTV microscopic]{
    \includegraphics[width=.23\linewidth, trim=45 443 562 510, clip]{../figures/planning-areas.jpeg}\label{fig:ctv-microscopic}
  }
  \subfigure[PTV area]{
    \includegraphics[width=.23\linewidth, trim=562 443 48 510, clip]{../figures/planning-areas.jpeg}\label{fig:ptv-area}
  }
  \caption{Simplified representation of the clinical target volumes. Figure~\ref{fig:gtv-macroscopic} shows the visible tumor, Figure~\ref{fig:ctv-microscopic} shows microscopic spread of the tumor, Figure~\ref{fig:ptv-area} shows area to account for short-term organ misalignment.}\label{fig:planning-areas}
\end{figure}

Oncologists use CT scans to draw clinical volumes by combining their knowledge about the particular cancer to determine target structures, organs-at-risk structures, and areas where the cancer will likely spread to~\cite{AMLART-data}. We provide a simplified representation in Figure~\ref{fig:planning-areas}.

The first area is the macroscopic delineated area of the visible tumour area. This Gross Target Volume (GTV) has a high probability of containing the tumour. Secondly, the Clinical Target Volume (CTV) is derived to account for potential microscopic spread. The CTV will be an area at least as big as the GTV with an optional margin surrounding it containing a 'rind' of non-zero probability of tumour spread. Lastly, the Primary Target Volume (PTV) contains residual geometric uncertainties and safety margins surrounding the CTV, ensuring the radiotherapy dose gets delivered to the CTV~\cite{tumor-delineation,defining-target-volumes,Lin2021-oz,personalised-PTV-strategies}. The PTV is a necessary extension of the CTV since geometric uncertainties are impossible and not advised to eliminate; after all, static scans are only estimations, subject to short-term organ misalignment, relative movement between structures of reference and tumours, partial volume effects and skewed anisotropic resolution~\cite{VANHERK200452}.

In parallel, the Oncologist constantly considers critical healthy tissue structures that need to be preserved during irradiation. These are referred to as organs-at-risk (ORs). In some specific circumstances, adding a margin analogous to the PTV margin around an OR is necessary to ensure that the organ cannot receive a higher-than-safe dose; this gives a planning organ at risk volume~\cite{defining-target-volumes}.

\subsection{International Guidelines}

The final volumes have no internationally agreed-upon guidelines, which leaves it up to the interpretation of the oncologists and Hospitals to use their heuristics when drawing areas. This time-consuming process has high variability, causing it to suffer significantly from inter and intra-observer variability~\cite{Lin2021-oz}. 

% However, the data provided has been standardised as a gold standard; see Section~\ref{sect:data}.

\subsection{Data Aquisition}

The Royal Marsden Hospital provides the dataset as a set of `Neuroimaging Informatics Technology Initiative' files (NIfTI)~\cite{file-formats}. It is a lightweight alternative to other formats such as DICOM (Digital Imaging and Communications in Medicine)~\cite{file-formats} and eliminates ambiguity from spatial orientation information~\cite{dicom-to-nifti-conversion}. Libraries exist for handling these files, such as SimpleITK~\cite{SimpleITK-paper}, which we use to read and manipulate the data in this project. % The library reads, manipulates, and handles the image as a set of points in a grid occupying a physical region in space defined by the metadata to remove ambiguity from the origin, size, spacing, and so on that might vary between patient scans.

The training data provides one hundred female patients that have been diagnosed with similar types of cervical cancer. Each patient comes with seven relevant segmentation classes which contribute to radiotherapy planning for cervical cancer. For reproducibility, all delineated anatomies were labelled consistently by the Oncologists to improve chances that an AI model can learn cervical cancer patterns~\cite{AMLART-data}. 

Finally, the dataset comes with ten held-out data items, which are patients with only the raw CT scan information without labels.

% \subsubsection{Notes}

% Some notes contain clinical observations about each of the 100 labelled data pairs~\cite{AMLART-data}. This small sample size of patients is also a good representation of the variability of data in the population. Because of the relatively small sample size, it is essential to be more acutely aware of the variability in the data.

% A common observation is that scans contain poor contrast. For patient 13, the note reads ``no contrast - hard to see LNs,'' which is information crucial to determining segmenting the Clinical Target Volume for Lymph Nodes (CTVn). Patients 9, 60, and 62 also have ``very large tumours''; sometimes, these even shift into uncommon areas, with ``sigmoid hanging into parametrium''. % GTV not visible?

% The notes help identify and diagnose some reasons for the model's poor performance in some cases, which may be due to the high variability between patients in quality and physiology.

\subsection{Delineation classes}

The clinicians at the Royal Marsden Hospital have provided segmentation labels for seven high-priority regions of interest (ROI). These are the Bladder, Anorectum, CTVn, CTVp, Parametrium, Uterus, and Vagina. The function of these anatomies is irrelevant to this project and is left to the reader to research further.

\subsubsection{Organs At Risk}\label{sec:data-organs-at-risk}

An organ at risk is an organ that, despite being healthy, is substantially likely to be within the PTV. Any areas created around the area should actively avoid these organs because overlapping with them risks complicating treatment and compromising the health of functioning organs. The key supplied anatomies at risk are the  Anorectum (Figure~\ref{fig:example-anorectum-axial}-\ref{fig:example-anorectum-sagittal}) and the Bladder (Figure~\ref{fig:example-bladder-axial}-\ref{fig:example-bladder-sagittal}).

\subsubsection{CTVp}\label{sec:data-CTVp}

The CTVp stands for the Primary Clinical Target Volume; see the example at Figure~\ref{fig:example-CTVp}. This is an area comprised from areas where there may be local microscopic spread (uterus, cervix, upper vagina, primary tumour)~\cite{AMLART-data}.

\subsubsection{CTVn}\label{sec:data-CTVn}

The CTVn stands for Nodal Clinical Target Volume; see the example at Figure~\ref{fig:example-CTVn}. This CTV surrounds areas that may contain microscopic spread to lymph nodes. It is drawn based on set margins around pelvic blood vessels and includes pelvic lymph nodes, common iliac lymph nodes and para-aortic lymph nodes~\cite{AMLART-data}.

Similarly to CTVp, this is a compound area with three groups of lymph nodes. In clinical practice, the number of these groups in the CTV varies in each patient, depending on the advanced disease. % TODO: reference notes? They say that depending on the development of the disease, different patients may get different extents of contouring of this structure.
However, in contrast to the CTVp, this area is drawn depending on the development of the disease.

\subsubsection{Parametrium, Vagina, and Uterus}\label{sec:data-Parametrium}

The Parametrium (or Paravagina) is the tissue surrounding the cervix/vagina at risk of local spread; see Figure~\ref{fig:example-Parametrium}. The Parametrium is drawn as a complete structure and edited back to the level of the vagina to be included~\cite{AMLART-data}.

Finally, the clinical significane of the Vagina and Uterus is to help define encapsulating structures like the CTVn (see Section~\ref{sec:rules}). Example for the Uterus and Vagina structures can be seen at Figure~\ref{fig:example-uterus-axial}-~\ref{fig:example-anorectum-sagittal} and Figure~\ref{fig:example-vagina-axial}-~\ref{fig:example-vagina-sagittal} respectively.

\subsection{Rules}\label{sec:rules}

Let us represent each organ anatomy as the first letter of its name, specifically: ($A$)norectum, ($B$)ladder, ($C$)ervix, ($P$)arametrium, ($U$)terus, ($V$)agina. Further, define:

\begin{enumerate}
  \item The CTVn and CTVp as $C_n$ and $C_p$ respectively
  \item The GTVn and GTVp as $G_n$ and $G_p$ respectively
  \item The Pelvic, Common and Para-aortic Lymph Node as $L_p$, $L_c$, and $L_{pa}$ respectively
\end{enumerate}

% \begin{minipage}{0.5\textwidth}
%   \subsubsection{Notation of Structures}
%   \begin{enumerate}
%     \item Let the Anorectum be denoted as $A$
%     \item Let the Bladder be denoted as $B$
%     \item Let the Cervix be denoted with $C$
%     \item Let the CTVn be denoted with $C_n$
%     \item Let the CTVp be denoted with $C_p$
%     \item Let the GTVp be denoted with $G_p$
%     \item Let the GTVn be denoted with $G_n$
%   \end{enumerate}
% \end{minipage}%
% \begin{minipage}{0.5\textwidth}

%   \begin{enumerate}
%     \setcounter{enumi}{7}
%     \item Let the Pelvic Lymph Node be denoted as $L_p$
%     \item Let the Common Iliac Lymph Node be denoted as $L_i$
%     \item Let the Para-aortic Lymph Node be denoted as $L_{pa}$
%     \item Let the Parametrium be denoted with $P$
%     \item Let the Uterus be denoted with $U$
%     \item Let the Vagina be denoted with $V$
%   \end{enumerate}

% \end{minipage}

\subsubsection{Relationship between Structures}

  \begin{enumerate}
    % \item If we want to talk about a specific patient, we should use the super-script notation to differentiate patients, e.g., $B^i$ for the Bladder of patient $i$.
    \item Let the overlap of two structures be denoted by the set intersect symbol $\cap$.
    \item Let the joint area of two structures be denoted by the set union symbol $\cup$.
  \end{enumerate}

  % \vspace{4em}

The top seven priority structures have been selected to identify and plan an area where radiotherapy should be used. With these structures, there are rules that the clinicians have outlined, they are quoted for clarification (these structures only refer to each independent patient):

\begin{enumerate}
  \item There should be no overlap between the CTVn, CTVp or Anorectum.

        \begin{equation}\label{eq:ctvn-ctvp-anorectum}
          \forall{i,j \in \{C_n, C_p, A\}}\text{ with } i \neq j, i \cap j = \emptyset
        \end{equation}

  \item The Parametrium may overlap with all of the other structures.

        \begin{equation}
          \forall i \in S, \quad (P \cap S \neq \emptyset) \vee (P \cap S = \emptyset), \quad \text{where } S = \{A, B, C, C_n, C_p, U, V\}
        \end{equation}

  \item The Bladder may overlap with the CTVn.

        \begin{equation}
          B \cap C_n \neq \emptyset \vee B \cap C_n = \emptyset\label{eq:ctvn}
        \end{equation}

  \item The CTVp is defined as a compound structure containing:

        \begin{equation}
          C_p = \overbrace{C \cup G_p}^{\text{High Risk CTV}} \cup \ \ U \cup V\label{eq:ctvp1}
        \end{equation}

  However, since we are never explicitly provided with the segmentation maps for the Cervix $C$ and the GTVp $G_p$, we cannot use as strong of a definition as above. Instead, we operate on the assumption that the union of the Uterus and Vagina is at least as big as the CTVp.

        \begin{equation}
          U \cup V \subseteq C_p\label{eq:ctvp2}
        \end{equation}

  \item The CTVn is defined as a compound structure containing:

        \begin{equation}
          C_n = G_n \cup L_i \cup L_p \cup L_{pa}
        \end{equation}

  Similarly, we are not provided segmentations for these areas, therefore, operating under no clinical knoweldge apart from the provided, cannot make any claims as to the composition of the CTVn.

\end{enumerate}

\subsection{Motivation in AI}

The medical sector has been a hotbed for AI research since  Convolutional Neural Networks (Section~\ref{sect:image-segmentation}) have been applied on medical image data by researchers. A branch of research dedicated itself to segmentation, which involves labelling individual pixels in the image according to which object or class they belong to. In dense classification, a model assigns every pixel to a specific class. Relevant to the direction of this project is determining the precise location and extent of organs or certain types of tissue, like ORs, CTV volumes, or other anatomies. 

The key objective of models trained for delineating target structures for this project is to see if an AI model can learn cervical cancer CTV pattern detection. The decision is complex as clinicians use information beyond the CT-imaging modality, such as how far along the tumour has progressed, and other clinical intuition to make proper judgements about the CTV volumes. Therefore, with this information missing from AI models, it is likely to misjudge target volumes, and a clinician will have to select which components of the CTV are required. However, a clinician will benefit from the time saved and improved consistency with the planning process if a trained model can produce the substructures required within the CTV that a clinician can review~\cite{AMLART-data}.

\section{Machine learning for image segmentation}\label{sect:machine-learning-for-image-segmentation}

Before the popularization of machine learning, algorithms used to be defined by strict and convoluted rules. These heuristically defined algorithms struggled to scale to complex problems and were often complicated or confusing to maintain. The typical task, however, does not warp easily into human intuition.

As a consequence of poor scalability, a new genre of algorithms began to emerge that fell into the classification of neural network models. Observations were rephrased and morphed into vectorised inputs, where each constituent of the vector represented a particular feature of the observation. For instance, the California Housing dataset~\cite{kelleypace1997} contains an input of 9 features (longitude, latitude, number of bathrooms, \dots) and one target feature (house price). After pre-processing and strategising, this input would be vectorised and fed into a network with several layers. Within each layer, sets of tunable parameters would optionally change the number of features and learn a relationship between parameters using weights until the $1 \times 9$ vector finally translates into a scalar value indicating the house price. After many repetitions of learning from examples, the model would learn an approximation to the complex objective. These models were termed Multi-Layered Perceptrons (MLPs).

\subsection{Image Segmentation}\label{sect:image-segmentation}

The current machine learning approach didn't work well with image data. Up until that point, rich structures such as images were neglected, and matrices of gray-scale images were mutilated into flat vectors. This approach was necessary to feed the flattened image representation through the MLP. The issue with vectorising an image is that it loses its spatial context-driven awareness. 

At the same time, J. Hull, sponsored by the United States Postal Service, published a Database for Handwritten Text Recognition with the incentive of providing an extensive dataset of images of characters of variable writing mediums, isolation, overlap, and neatness to aid research efforts in developing accurate digit classification algorithms~\cite{JJHull1994}. Yann LeCunn et al.~\cite{Lenet1998} used this database to propose the first Convolutional Neural Network (CNN) for image processing, which preserved the input in its 2D glory by applying convolutions. The fundamental Convolutional Filter is displayed in Figure~\ref{fig:sobel}.

\begin{figure}[H]
  \centering
  \includegraphics[width=1\linewidth]{../figures/sobel.png}
  \caption{Example application of a convolution. From left to right, the input image with a region outlined in a red box, the boxed region magnified with a convolutional (Sobel) filter being applied to a part of the magnified region, and lastly the output after the filter has passed over the entire image. The output represents a new feature map encoding features of the original (input) feature.}\label{fig:sobel}
\end{figure}

Convolutional layers are rectangular blocks that are recipes for translating an image (alternatively known as the input feature). The algorithm centres the block over a specific pixel and uses a square radius of neighbouring pixels. It multiplies and sums the pixels along the corresponding pixel positions according to the recipe to produce a transformed resulting pixel that encodes the reference pixel's information and the surrounding receptive field around it. Figure~\ref{fig:sobel} demonstrates this concept. Specifically the middle tile, which shows an example $3 \times 3$ convolutional filter being applied to a zoomed in part of the image. This filter slides across the entire image and encodes it which produces the output on the right of Figure~\ref{fig:sobel}. 

The CNN operates by having multiple learnable convolutional filters stacked on top of each other. The values within the square convolutional filter are \textit{learned} during training; the values learnt are those that, in combination with the layers before and after, encode the image's features according to the training objective the best. When stacked with many other convolutional filters that piggyback off the encoded features produced by filters before it, this allows for dense feature representations that encode the entire image.

\begin{figure}
  \centering
  \subfigure{\includegraphics[width=.49\linewidth, trim=0 230px 0 0, clip]{../figures/fcnupsample.png}}
  \subfigure{\includegraphics[width=.49\linewidth, trim=0 33px 0 200px, clip]{../figures/fcnupsample.png}}
  \caption{Comparison of upsampling a base image using FCN~\cite{fully-CNNs-for-semantic-segmentation} and the VGG-16-based DeconvNet~\cite{noh2015learning, simonyan2014very} architectures.}\label{fig:fcn-vs-deconvnet}
\end{figure}

This first convolutional network spawned a vicious flurry of convolutional architectures, which followed in their footsteps. The Fully Connected Neural Network (FCN) adapted the architecture used by LeCunn et al.~\cite{Lenet1998} for segmentation applications. Previously, convolutions would reduce input image feature vectors into non-spatial classification outputs. However, this paper `convolutionalized' the pipeline to provide a heatmap of segmented objects within the image~\cite{fully-CNNs-for-semantic-segmentation}. The heatmap would describe in a 2D feature the location of each class.

Trivial upsampling through de-convolutional layers allows the heatmap to translate back to the original size. This process would produce a largely inaccurate segmentation with much room for improvement. Therefore, similarly to learning the downsampling, the model learnt to upsample low-level heatmap representations~\cite{noh2015learning}. This way, the deconvolutional network also became ``a key component for precise object segmentation'', which improved the base upsampling provided by the FCN. This conclusion is shown in Figure~\ref{fig:fcn-vs-deconvnet}.

The strategy of downsampling and upsampling for image segmentation is a common theme amongst many segmentation architectures.

\subsection{U-Net}

\begin{figure}[H]
  \centering
  \includegraphics[width=.8\linewidth, trim=0 200px 0 0, clip]{../figures/u-net.png}
  \caption{The U-Net architecture, with contractive side on the left, and expansive side on the right. The feature map follows the arrows and multiplies the number of feature maps twice at each contraction, and halves at each expansion~\cite{U-Net}.}\label{fig:unet}
\end{figure}


Simultaneously, a unique architecture was under development. This architecture mirrored the hourglass structure of contracting and upsampling sections of the FCN, only this time, the illustration of its architecture led to its name, the U-Net~\cite{U-Net}.

The U-Net similarly consists of a contractive and expansive side, with the added feature of so-called `skip-connections'. As shown in Figure~\ref{fig:unet}, at each stage of the contracting/expansive side, these copy operations help localise high-resolution features from the contracting path~\cite{U-Net}. The network yielded great results for a few training images and more precise segmentations.

% TODO: skip connections glue the output of previous layer on

The U-Net was now very close to being a staple choice in biomedical data for anatomy segmentation. A limitation of its current implementation is that slices along the third dimension would most certainly contain contextual information that would influence the decision of the current design. Therefore, an identical network topology with extended 3D convolutions was proposed by \c{C}i{\c{c}}ek et al.~\cite{DBLP:journals/corr/CicekALBR16}.

\subsection{nnUNet}\label{sect:nnunet}

The nnUNet is closely related to the U-Net architecture. However, nnUNet's ability to adapt to the data improves this model performance over the vanilla implementation.

Until now, architectures have operated on data pre-processed to a particular expected distribution and configured to deal with a specific problem. For instance, out-of-the-box configuration has stringent input size requirements, which require image resizing. Furthermore, 3D images commonly produce heterogeneous voxel spacing depending on the parameters chosen by the clinician or the machine from which they were produced~\cite{nnunet}. The number of moving parts often leaves a unidirectional dependency on the data depending on the architecture. 

\subsubsection{Automated Method Configuration}

Therefore, the nnUNet analyses the fingerprint of the dataset and the device to deliver a tailored experience and force a more codependent relationship; now, the architecture depends on the data and the data is pre-processed to conform to the network~\cite{nnunet}. Furthermore, hardware restrictions mean networks may be inaccessible to those with worse specifications or, at the other end of the spectrum, may underutilize powerful computation still available~\cite{nnunet}; the nnUNet analyses GPU constraints used to influence batch sizes and more~\cite{nnunet-git-paper}.

The automated method configuration is classified into three categories. A dataset fingerprint extracts training data distributions such as shape, spacing and intensity distributions. Rule-based parameters estimate the most common robust parameters for resampling and normalization. Finally, the Empirical Parameters learn parameters, such as ensemble selection, which is not derivable from the dataset fingerprint.

\subsubsection{Critical Review of the nnUNet}

A review of segmentation methods in 2024 reviewed some further developments in segmentation models and found that the convolution-based U-Net architectures continued to outperform Attention-based or Mamba-based approaches six years after the initial publication of the self-configuring network~\cite{isensee2024nnunet}. Isensee et al. concluded that there was a significant mischaracterisation of proclaimed improvements in new strategies such as transformers. Claims of performance improvements over the nnUNet were reviewed through the control of validation datasets and removals of baseline tampering, which demonstrated the convolution-based performance on datasets with low statistical intra-method standard deviation~\cite{isensee2024nnunet}.

The continued performance dominance gives the nnUNet a good foundation for being used as a baseline model for all datasets.

\subsection{TotalSegmentator}

TotalSegmentator is a tool based around the nnUNet. TotalSegmentator is pre-trained on 1204 CT examinations to provide plans to segment 104 anatomical structures. The anatomies selected included apparent structures such as skeletal structures, gastrointestinal organs and other major organs. The training data contained many CT images, with differences in slice thickness, resolution, and contrast phase~\cite{totalsegmentor-paper}. However, it is important to note that 60\% of the scans occurred in a contrast-enhanced environment, which plays a role in how obvious delineations are during scanning, a scanning detail that was omitted during the training data collection for this research project. Furthermore, only an estimated 10\% of the data collected from this model contained relevant studies for the abdomen and pelvic areas.

From the segmented organs that total segmentator provides, only the Bladder overlapped with the organs that were of interest in this study.

\subsection{UniverSeg}\label{sect:universeg}

Convolutional architectures like those discussed above utilize many-shot learning (Section~\ref{sect:many-shot-learning}). However, models trained on segmenting a target domain (e.g., bone delineation) do not transfer to other domains (e.g., organ segmentation) without fine-tuning.

\begin{figure}[H]
  \centering
  \includegraphics[width=.5\linewidth]{../figures/universeg.png}
  \caption{The UniverSeg architecture~\cite{universeg}. Diagram illustrates the freezing of model parameters while providing a support set of images which follows the query image through the segmentation pipeline.}\label{fig:universeg}
\end{figure}

Butoi et al. present UniverSeg, a model that breaks away from the traditional approach; instead of training a model on a single task (e.g. bone delineation) and freezing parameters during inference, this architecture uses a support set of images to provide a practical few-shot approach to inferring segmentations from input images. Figure~\ref{fig:universeg} shows an example of this efficient querying, which manifests itself in a U-Net architecture where the support set passes through the network along the query to influence the final segmentation. This way, Butoi et al. attempt to provide segmentation on a target image based on examples of other samples with the same anatomy contoured in a selection of other images. Therefore, the model can learn to segment both bone and organ segmentation tasks without the need for fine-tuning~\cite{universeg}.

This model operates on 2D slices of images and directly avoids the finetuning argument for medical imaging. They argue that finetuning can be unhelpful due to the differences between medical domains, features, and data fingerprints. As such, UniverSeg avoids significant retaining for each subtask.

% Fully transferable, no retraining, few shot learning

% This is especially problematic in the medical domain where clinical researchers or other scientists are constantly defining new segmentation tasks driven by evolving populations, and scientific and clinical goals. To solve these problems they need to either train models from scratch or fine-tune existing models.

% Fine-tuning models trained on the natural image domain can be unhelpful in the medical domain [ 82 ], likely due to the differences in data sizes, features, and task specifications between domains, and importantly still requires substantial retraining. 

% Universeg avoids substatial retraining for each subtask

\subsection{SAM}\label{sect:sam}

\begin{figure}
  \centering
  \includegraphics[width=1\linewidth]{../figures/SAM.png}
  \caption{The SAM model involves a transformer architecture that embeds points and bounding boxes into a promptable encoding, which is used in tandem with the image encoding to produce the most likely segmentation of the described area~\cite{SAM}.}\label{fig:sam}
\end{figure}

Advances in NLP, with attention-based mechanisms, have questioned whether convolutional-based methods like those above are the best approach for segmentation. Transformers from NLP were adapted to form the Vision Transformer (ViT)~\cite{ViT}. This model views the image as a grid of tokens; in the original paper, Dosovitskiy et al. separate the image into a grid of patches and read in these grid cells as individual tokens. The tokens pass through the attention mechanism as with NLP and into a classification mechanism~\cite{ViT}. 

% TODO search and replace normalize -> normalise
The SAM (Segment Anything Model) model implemented a modification of the transformer architecture~\cite{SAM} as seen in Figure~\ref{fig:sam}. However, the task was reformulated as a promotable segmentation problem to allow for zero-shot generalisation (the model can generalise to unseen examples with no re-training or fine-tuning). As seen in Figure~\ref{fig:sam}, the model inputs an image along with either points in the image or boxes. This reduces the search space SAM has to perform to segment an object into an area or a set of points.

\subsection{MedSAM}\label{sect:medsam}

\begin{figure}[H]
  \centering
  \includegraphics[width=1\linewidth]{../figures/sam-performance.png}
  \caption{Performance of a SAM model across a reference nnUNet baseline when applied to different datasets. The evaluation was considered for semantic, point and box prompts.~\cite{he2023computervision}.}\label{fig:sam-performance}
\end{figure}

SAM trains its network on a collection of natural images, not medical images such as CT and MRI scans. Extensive stress tests performed on SAM concluded that SAM's out-of-the-box promotable segmentation tool had good baseline performance on large visible objects~\cite{deng2023segment} but required exact prompt segmentation, making it inaccessible to automated contouring. Regarding the number of points required to make a sensible prediction, SAM quantitatively underperformed a nnUNet baseline, with qualitative evaluation showing fuzzy boundaries in medical contexts~\cite{hu2023sam}. Finally, Figure~\ref{fig:sam-performance} shows the performance of SAM against an nnUNet baseline across a set of datasets and imaging modalities~\cite{he2023computervision} which concludes that SAM never outperformed the nnUNet baseline when trained on the 11M natural images~\cite{SAM}. 

Therefore, Ma et al. enhanced the architecture provided by SAM by training it on a dataset of nearly 500k CT scan test examples to train a model for medical images, named MedSAM~\cite{Ma2024}. Ma et al. decided to keep close to its original despite medical images being 3-dimensional in CT and MRI scans because of ``enhanced flexibility and adaptability'' where slices along an axis substitute 3D scans~\cite{Ma2024}. This model demonstrates an improvement over SAM, nnUNet, and Deepmedic models when MedSAM bounding boxes extracted from the ground truth prompt the model~\cite{Ma2024}.

\section{Transfer Learning}

Transfer Learning involves using a model trained on a large dataset. The pre-trained model serves as the starting point for a second task where data acquisition is limited or improbable. Transfer learning is successful because, in the early layers of a model, it typically learns very low-level features. At this scale, the objective of the original domain does not matter; regardless of the initialisation, a model working on a similar problem will inevitably learn similar low-level features. 

Arguably, the universality of the parameters learnt in a model with prosperous access to data will have richer and better patterns that another with less data did not have enough information to learn~\cite{deep-learning-book, survey-on-transfer-learning}.

Transfer Learning has the potential to improve initial performance using only the transferred knowledge before any further learning begins, improve the time it takes to thoroughly learn the target task given the transferred knowledge, and improve the final performance all when compared to initial benchmarks without transfer~\cite{torrey-handbook}. Medical contexts have already applied Transfer Learning, which reportedly improved weight initialisation for 332 abdominal liver CT scans and resulted in faster convergence, providing a more robust representation~\cite{liver-lesion-via-transfer-learning}.

Transfer Learning has been seen to prevent overfitting in domains where data volume is low and where generality without overfitting is hard to come by. The prevention is because the model has already learnt features likely to be helpful in the second task~\cite{geeks-transfer-learning}. Overfitting may still occurr if the model is fine-tuned too much on the second task, as it may `learn task-specific features that do not generalise well to new data'~\cite{geeks-transfer-learning}.

\subsection{Mathematical Definition}

Transfer Learning can be formalised mathematically~\cite{concise-review-of-transfer-learning, survey-on-transfer-learning}.

\begin{itemize}
  \item Define the starting domain the model trains on initially as $\mathcal{D} = \{\mathcal{X}, P(X)\}$ with feature space $\mathcal{X}$ and a marginal probability distribution $P(X)$. $X$ is defined as an instance set, and $X = \{x_1, x_2, \cdots, x_n\} \in \mathcal{X}$.
  
  \item The original task attempts to predict data from domain $\mathcal{D}$ through a mapping $\mathcal{T} = \{\mathcal{Y}, f(\cdot)\}$ composed of a label space $\mathcal{Y}$ and an objective predictive function $f(\cdot)$. Given a domain $\mathcal{D} = \{\mathcal{X}, P(X)\}$ the sample data consists of pairs $\{x_i, y_i\}$ where $x_i \in X$ and $y_i \in \mathcal{Y}$. The objective function $f$ is supposed to learn from sample data to predict the corresponding label for the new instances. $f$ can be rewritten as $f(x)=P(y|x)$.
  
  \item The transfer learning task involves a source domain $\mathcal{D}_S$ with corresponding source tasks $\mathcal{T}_S$, and a target domain $\mathcal{D}_T$ with corresponding target task $\mathcal{T}_T$. The goal is to transfer the related knowledge to boost the performance of the target predictive function $f_T(\cdot)$.
\end{itemize}

\subsection{Shot based learning}

\begin{figure}[H]
  \centering
  \raisebox{-0.5\height}{
    \subfigure[Zero-shot]{
      \includegraphics[width=.4\linewidth]{../figures/zero-shot-learning.png}\label{fig:zero-shot}
    }
  }
  \raisebox{-0.5\height}{
    \subfigure[Few-shot]{
      \includegraphics[width=.4\linewidth]{../figures/few-shot-learning.png}\label{fig:few-shot}
    }
  }
  \caption{Shot based learning captured in the natural language context~\cite{openaishotbasedlearning}}
\end{figure}

The transfer of models is dependent on the strategy the model uses to make predictions. Shot-based learning is a substantial factor. This describes the model's exposure to data in domains and its ability to handle cases it has never seen before. Thus, shot-based learning can be separated into meta-learning tasks or multitask learning. 

Instead of learning underlying patterns, meta-learning models learn to \textit{learn} the algorithm itself. This allows it to generalize tasks with few labelled examples of new or rare cases. Otherwise, multitask learning makes predictions for a particular set of tasks~\cite{deep-learning-book}.

\subsubsection{Zero-shot Learning}\label{sect:zero-shot-learning}

Zero-shot models (Figure~\ref{fig:zero-shot}) can generalize to unseen tasks. An example of a model targeted for zero-shot transferability is the SAM model (Section~\ref{sect:sam}).

\subsubsection{Few-shot Learning}\label{sect:few-shot-learning}

Few-shot models (Figure~\ref{fig:few-shot}) can generalize to unseen tasks with a few examples. The UniverSeg model is the target model for few-shot transferability (Section~\ref{sect:universeg}).

\subsubsection{Many-shot Learning}\label{sect:many-shot-learning}

Many-shot models can generalize to unseen tasks with many examples. The nnUNet based models such as the TotalSegmentator is the target model for many-shot transferability (Section~\ref{sect:nnunet}).

%%%%%%%%%%%%%%%%%%%%%%%%%%%%%%%%%%%%
\chapter{Methodology}\label{sect:methodology}

Improvements over the transfer from models trained on other domains could hypothetically allow the segmentation of delineated areas more accurately. To investigate this claim, we iterate over the three types of shot learning, zero-shot, few-shot, and many-shot learning, to determine the effectiveness of transfer learning in the radiotherapy domain.

\section{Transfer Strategy}

Transfer learning involves obtaining a pre-trained "base" model, identifying transfer layers, and fine-tuning the remaining layers. Key considerations include preventing overfitting on smaller datasets and making decisive adjustments to learning rates during the fine-tuning phase.

\begin{enumerate}
  \item Obtain a pre-trained ``base'' model trained on extensive data which identifies general features and patterns relevant to the target domain.
  
  \item Identify the transfer layers. These layers capture general information relevant to both the new and previous tasks. The selected layers are 'frozen' during training. This means parameters are not changed, preserving the low-level learnt feature functions. 
  
  The number of frozen layers depends on how much inheritance is required from the pre-trained model. For unrelated domains, like car vs. face detection, only low-level features should be transferred. However, the amount of information transferred in related domains is more generous, and the quantity is tunable.
  
  \item Fine-tune and retrain the remaining layers. The goal is to preserve the knowledge from the pre-training while enabling the model to modify its parameters to suit the demands of the current assignment better~\cite{geeks-transfer-learning}.
  
  For smaller datasets, there may be an issue of overfitting because the number of available samples to train on has decreased. Thus, we must set the learning rate to be low; when fine-tuning a new model you want to readjust the pre-trained weights, and if the learning rate is too high, the model may rapidly overfit~\cite{geeks-transfer-learning}. Furthermore, this is done for a low number of iterations for the same reasons~\cite{deep-learning-book}.
\end{enumerate}

\section{Baseline -- nnUNet}

The provided examples of the objective are enough to train an nnUNet model from scratch. It has been shown that in many biomedical applications, ``only very few images are required to train a network that generalises reasonably well''~\cite{DBLP:journals/corr/CicekALBR16}. Therefore, we first train a baseline nnUNet model on the provided examples to establish a benchmark for the transfer models.

\subsection{Preprocessing}

Prior to training, necessary normalization must take place to standardize each input. Firstly, CT scans can produce results for different spacings, resolutions, and dimensions. Because the model samples data in batches, the batch properties must align within the model's body. Therefore, the fingerprint taken by the preprocessing pipeline resamples input data towards the median dimension. 

The traditional method of normalization involves normalizing the entire image corpus. However, this method does not take into account the skew that might affect the outcome. This is because the background value occurs most frequently, and artifacts such as metal cause outlier peaks. As a result, traditional normalization could overlook important tissues as it attempts to treat background and outlier values equally with foreground values.

Therefore, the nnUNet avoids this complication by processing voxel properties encased by the ground truth segmentation. The values are clipped to their 0.5 and 99.5 percentile, followed by traditional normalization with the mean and standard deviation. This way, the target structure properties remain relative to their original, and the background label conforms to the normalization inspired by the region of interest.

\subsection{Separate Training}

The default strategy is to consider each anatomy separately and attempt to learn segmentation patterns without considering constraints mentioned in Section~\ref{sec:rules}. These models can be used in an ensemble system to produce thorough segmentations of the target volumes, oblivious to clinical constraints. 

We set up the learning pipeline to follow a five-fold cross-validation process. Each fold involves splitting the data into five subsets. Then, the model is trained on four subsets and validated on the fifth subset. The process repeats for each of the five subsets, with each subset used exactly once as the validation set. Cross-validation helps assess the model's performance and generalize it to new data. Ultimately, the model with the best validation performance represents the anatomy.

\section{Many-shot Transfer -- TotalSegmentator}

We kickstart the transfer discussion by evaluating the transfer of many-shot models. The candidate model for this is the TotalSegmentator model, which is entirely based on the default implementation of the nnUNet. 

When transferring information, many-shot models are the most obvious choice. These models have the potential to increase the amount of helpful information that can be used to segment anatomies in the target domain by transferring the original task. The hypothesis is that many-shot transfer models will improve segmentations for clear organ delineations. Additionally, anatomies that have already been segmented (such as the bladder) will fine-tune to become more accurate.

\subsection{Separate Training}

Like the nnUNet, we apply the default fine-tuning strategy to a pre-trained TotalSegmentator model. TotalSegmentator has 23 separate nnUNet pre-trained models on different sub-tasks in the anatomical segmentation~\cite{totalsegmentor-git}. These individual models segment structures like the vertebrae, cardiac muscles, rubs, and lung vessels. For this destination delineation task, we chose the 'organ' model trained on a withheld dataset of 1200 CT scans. The only anatomy shared between the two domains is the Bladder which will hypothetically provide a performance improvement for this anatomy due to the additional thousand examples of the bladder.

\subsection{Region Based Training}

\begin{figure}[H]
  \centering
  \subfigure[Anorectum]{\includegraphics[width=.24\linewidth, trim=0 0 4400px 110px, clip]{../figures/check_if_worked_axis_1.png}}
  \subfigure[Bladder]{\includegraphics[width=.24\linewidth, trim=630px 0 3770px 110px, clip]{../figures/check_if_worked_axis_1.png}}
  \subfigure[CTVn]{\includegraphics[width=.24\linewidth, trim=1260px 0 3140px 110px, clip]{../figures/check_if_worked_axis_1.png}}
  \subfigure[CTVp]{\includegraphics[width=.24\linewidth, trim=1890px 0 2510px 110px, clip]{../figures/check_if_worked_axis_1.png}}
  \subfigure[Parametrium]{\includegraphics[width=.24\linewidth, trim=2520px 0 1880px 110px, clip]{../figures/check_if_worked_axis_1.png}}
  \subfigure[Uterus]{\includegraphics[width=.24\linewidth, trim=3150px 0 1250px 110px, clip]{../figures/check_if_worked_axis_1.png}}
  \subfigure[Vagina]{\includegraphics[width=.24\linewidth, trim=3780px 0 620px 110px, clip]{../figures/check_if_worked_axis_1.png}}
  \subfigure[Combined Labels]{\includegraphics[width=.24\linewidth, trim=4430px 0 0px 140px, clip]{../figures/check_if_worked_axis_1.png}\label{fig:combined-labels-region-based}}
  
  \includegraphics[width=\linewidth]{../figures/check_if_worked_axis_1_legend.png}
  
  \caption{New IDs generated for a single slice as a consequence of practical experience show non-conformity to rules in Section~\ref{sec:rules}.}\label{fig:check-if-worked-axis-1}
\end{figure}

The natural extension of separate training is to consider training each class simultaneously. Region-based training is a noninvasive method offered by the nnUNet. This training style combines the 3D segmentation maps for each of the seven classes into one 3D segmentation by introducing new IDs wherever there is a new overlap. An example slice can be seen in Figure~\ref{fig:check-if-worked-axis-1}.

For the purposes of evaluation, the following methods will be applied to the baseline nnUNet model to assess the success.


\subsubsection{No Rule Enforcement}

A sensible and calibrated set of ids could be drawn based on the rules in Section~\ref{sec:rules}. However, practical experience and the recalcitrant and challenging nature of defining logical expressions to generalize something as complex as organs and something as fuzzy as microscopic cancer spread mean that, in practice, these rules cannot be implemented strictly.

Please refer to Equation~\ref{eq:ctvn-ctvp-anorectum} ``There should be no overlap between the CTVn, CTVp or Anorectum''. We see marginal overlap between the structures in Figure~\ref{fig:combined-labels-region-based} in ids 55, 94. These are almost not visible on the figure, but captured by the legend. 

\begin{figure}[H]
  \centering
  \includegraphics[width=.7\linewidth]{../figures/percentage_of_illegal_overlap.png}
  \caption{Total frequency and percentage overlap of non-zero occurrences of `illegal' overlap between the CTVn, CTVp, and Anorectum.}\label{fig:illegal_overlap}
\end{figure}

When considering the total sample set, the overlap makes up a minimal percentage of the total volume with non-background label classification. Take, for instance, Figure~\ref{fig:illegal_overlap}, which shows all cases of `illegal' overlap found within the samples. The most frequent culprit overlap that breaks the condition is between the CTVn, CTVp and the Uterus, which makes up the most significant proportion of `illegal' overlap. The Uterus is a necessary subset of the CTVp, which implies that this segmentation of CTVp does not include microscopic spread, including in the Cervix and GTVp. By eliminating this style of error from the model, it might be possible to obtain more general segmentations that align with the clinical team's rules.

\subsubsection{With Rule Enforcement}

We implement the rules with the hypothesis of generalisation despite the rules not strictly followed in practice. Therefore, an augmentation of the nnUNet trainer is necessary to capture this result. Specifically, we introduce an additional loss component dubbed `custom\_loss' during training. This custom loss parameter implements two of the many rules that the Royal Marsden Hospital clinical staff provided. The two rules are thoroughly discussed in Section~\ref{sec:rules} and are Equation~\ref{eq:ctvn-ctvp-anorectum} and Equation~\ref{eq:ctvp2}. 

To consistently include the loss parameter, we have two sub-loss objectives operating on dice loss (Section~\ref{sect:spatial-overlap-based}). The first captures Equation~\ref{eq:ctvn-ctvp-anorectum}, which focuses on the strict requirement that the CTVn, the CTVp, and the anorectum do not overlap. The loss term is minimal when the dice score is the lowest, indicating minimal overlap; thus, to calculate this result on the intersection, the implementation of vanilla dice loss applies to the rule's three permutations independently. Secondly, Equation~\ref{eq:ctvp2} insights more overlap between the vagina and uterus with the CTVp, which is equivalent to 1 minus the previous learning objective. As a hyperparameter, we attribute a weight of 0.3 to this term, while other loss terms (dice and binary cross-entropy) are assigned weights of 1 to keep the original objective function reasonably unchanged.

We hypothesise that although the rule-free model attempts to approximate the ground truth, which may call for marginal overlap, with the rules in place, this may generalise better over other cases.

\section{Few-shot Transfer -- UniverSeg}

\begin{figure}
  \centering
  \includegraphics[width=.8\linewidth]{../../research/source/code/UniverSeg/start_end_distribution_axis_0_anorectum.png}
  \includegraphics[width=\linewidth, trim=0 0 340px 0, clip]{../../research/source/code/UniverSeg/support_anorectum_axis_0.png}
  % \subfigure[]{\includegraphics[width=.45\linewidth]{../../research/source/code/UniverSeg/start_end_distribution_axis_1_anorectum.png}}
  % \subfigure[]{\includegraphics[width=.45\linewidth]{../../research/source/code/UniverSeg/start_end_distribution_axis_2_anorectum.png}}
  \caption{The distribution of the start and ending slices of the tumor in a normalized batch of images across axes 0 (axial dimension) and the corresponding sampled support extracted.}\label{fig:start-end-distribution-universeg}
\end{figure}

UniverSeg operates on 2-dimensional data. As a result, to provide automatic radiotherapy planning volumes, it is essential to provide slices for the model as input alongside a sufficient support size of similar slices. 

\subsection{Preprocessing}

Images are normalised to match UniverSeg's training properties. Specifically, image intensities are clipped to the range $[-500, 1000]$ and normalised to be between $[0,1]$ and finally scaled down to a $128 \times 128$ resolution~\cite{universeg}. Slice pre-processing is repeated along each axis to evaluate the effectiveness of one slice over another. 

\subsection{Selecting support}

The most screaming disadvantage of such a model when applied to a three-dimensional unseen segmentation task is tumour localisation; we do not know where the tumour starts or ends. The tumour's location can only be determined by referring to assisting models that provide estimates or make assumptions about tumour location based on previous examples. In a secluded environment where only one's model transferability is assessed, we instead analyse the properties of the tumour from the examples.

We begin by analysing the tumour locations along each axis. The intuition is that the pre-processed images of equal dimensions and spacings should contain the tumour at approximately the same location. Figure~\ref{fig:start-end-distribution-universeg} shows the tumour's distribution of start and end slices and the corresponding sampled support set.

By sampling a support set concerning slice positions, we include an approximation of a binary classification of `does this slice contain a tumour or not'; if for slice n there is a 20\% chance that the support contains a tumour, then we can also assume that for an unseen patient, this chance is also the same. However, this does not generalise to boundary cases where tumours or organs may begin sooner or later than all others, which may cause a misclassification.

\section{Zero-shot Transfer -- MedSAM}

MedSAM is the final algorithm in the shot-learning strategies that will be covered. Specifically, MedSAM is being evaluated on the provided dataset, both with and without transfer, to test the claim of transferability.

Zero-shot learning, a promising feature of MedSAM, could potentially enable it to handle tasks it has never encountered before. However, the assertion that it can seamlessly transfer without refinements is a bit of a stretch. This is because certain contours, like the CTVn, delineate tiny tumor spread throughout the lymph node system. Therefore, it is unlikely that MedSAM, trained in such cases, would be used for these specific tasks. It would typically be used for more straightforward volume delineations, such as organs.

\subsection{Preprocessing}

MedSAM takes two-dimensional images as input, much like UniverSeg. Therefore, we similarly preprocess the data to a resolution of $1024 \times 1024$ along each slice where the ground truth is present. We clip the ranges of the CT scan to Hounsfield units centred around the 40 value, with a window radius of 400 units and later normalized to the range of $[0,1]$. In contrast to UniverSeg, we resize a slice after selecting the index instead of resizing the whole image. Thus, we do not introduce any anisotropic uncertainties along our sampling axis.

\subsection{Point based transfer}

We first experimented with the MedSAM architecture to test the effectiveness of prompting on points. We formulated this task into two categories: a sparse and dense point experiment. 

\begin{figure}[H]
  \centering
  \raisebox{-0.5\height}{\subfigure[Sparse point prompt]{\includegraphics[width=.45\linewidth, trim=850 850 0 0, clip]{../../research/source/code/SAM_Med/3_train_point_prompt/point_batch_example.png}}}
  \raisebox{-0.5\height}{\subfigure[Dense point prompt]{\includegraphics[width=.45\linewidth, trim=0 0 850 850, clip]{../../research/source/code/SAM_Med/3_train_point_prompt/point_dense_batch_example.png}}}
  \caption{The two strategies for sampling structures in the MedSAM model.}\label{fig:point-prompt}
\end{figure}

The two strategies are shown in Figure~\ref{fig:point-prompt}. The sparse point prompt is a single point in the image, while the dense point prompt is a set of $n$ points randomly sampled from the ground truth segmentation.

The hypothesis is that point prompts may work well with structures such as the bladder which are trivial to identify. However, for the CTV, the point prompt may not segment the full extent of the structure.

Points are sampled with a higher probability of being towards the centre of the segmented area by calculating the Euclidean distances of all points from the centroid. This heuristic dispels any ambiguity that may be caused when selecting points closer towards the outline of the segmentation, which may occur when boundaries of different objects bleed into one another, causing a higher error.

\subsection{Box based transfer}

A box prompted solution will define the area where the tumour is likely to be. Box prompts for training are obtained from the ground truth, with a margin surrounding the box. 

For training, only one box may be fed into the model for inference. Therefore, we select a random structure from the boxes encompassing the tumour for a given slice. Figure~\ref{fig:box-prompt-sam-selection} demonstrates the tumour box selection at training time. 

\begin{figure}[H]
  \centering
  \subfigure[Boxed ground truth image]{
    {\includegraphics[width=.3\linewidth, trim=40 30 610 30, clip]{../../research/source/code/SAM_Med/4_train_box_prompt/select_top_n_largest_structures.png}}
  }
  \subfigure[Selected tumor boxed]{
    {\includegraphics[width=.3\linewidth, trim=340 30 310 30, clip]{../../research/source/code/SAM_Med/4_train_box_prompt/select_top_n_largest_structures.png}}
  }
  \subfigure[Ground truth augmentation]{
    {\includegraphics[width=.3\linewidth, trim=630 30 20 30, clip]{../../research/source/code/SAM_Med/4_train_box_prompt/select_top_n_largest_structures.png}}
  }
  \caption{The process of selection bounding box for a given boxed tumor with an augmented ground truth used at training time for the MedSAM bax-based prompt model.}\label{fig:box-prompt-sam-selection}
\end{figure}

\subsection{Evaluation}

During training, the test set is divided by ID to ensure that a portion of an image is not both trained and validated. This way, the model's transferability is not questioned because it is trained on a disjoint set of patients; otherwise, quantitative measurements would skew and not represent the model's adaptability to unseen data.

Secondly, the MedSAM architecture selects points and boxes for inference from the ground truth. Because MedSAM is an interactive tool, its application will likely adapt as an assisting tool for clinicians. However, for unseen examples, the concept of auto planning is uncertain. The authors of the paper measure performance according to the ground truth~\cite{SAM, Ma2024}. Refining the model pipeline to include bounding box predictions extracted from a baseline like nnUNet is possible, where MedSAM acts as a refinement to an already reasonably accurate guess. This allows for noise to be permeated into the model, thus allowing for a better auto-contour planning pipeline.

\section{Quantiative Evaluation of Segmentation}\label{sect:performance-evaluation}

Calculating the difference between the provided labelled data would be one way to determine if a contour can be used in a clinical context. However, we have different ways to evaluate this measure in a delineation context.

If we were attempting to fit a model onto a line in 2D space, the performance of our model would be the total minimum distance between each point and the prediction. Our objective would be to drive the model's distance metric as close to 0 without overfitting. Here, the points act as a 'ground truth', alternatively referred to as the gold standard, which represents the actual measured value.

The reasoning above extends to 3D and 2D in a segmentation context with variants to measure other quantities, like the minimum distance between prediction and truth or the extent of volume overlap between the two. These are examples of geometric measures, which Mackay et al. has found to be the most popular measure in segmentation tasks~\cite{review-metrics}.

\subsection{Classification Based}\label{sect:classification-based}

Assesses if voxels within and outside the auto-contour have been correctly labelled~\cite{review-metrics}. To begin, we define 'positive' to mean that the voxel selected indeed needs radiotherapy treatment and 'negative' to mean that the voxel classifies as healthy.

A standard measure of classification is accuracy. It measures the total number of correct predictions vs. the total predictions it made. However, more than this measure is needed to fully capture a model's bias because it does not tell the whole story with class-imbalanced data when there is no even number between positive and negative labels.

\begin{equation*}
 \text{Accuracy} = \frac{TP + TN}{TP + TN + FP + FN}
\end{equation*}

Better measures are Precision and Recall scores. The Precision (also known as the Positive Predictive Value~\cite{evaluation-metrics}) measures the proportion of successfully correct predictions. The Recall (also known as True Positive Rate~\cite{evaluation-metrics}), on the other hand, ``measures the portion of positive voxels in the ground truth that is also identified as positive by the segmentation being evaluated''.

\begin{equation*}
 \text{Precision} = \frac{TP}{TP+FP}, \quad \text{Recall} = \frac{TP}{TP+FN}
\end{equation*}

\subsection{Spatial Overlap Based}\label{sect:spatial-overlap-based}

Similarly to classification-based metrics in Section~\ref{sect:classification-based}, an overlap-based metric measures the extent of overlap between an auto-contour and a reference structure~\cite{review-metrics}.

The scores above combined into a more general score $F_\beta$ to give

\begin{equation*}
 \text{F}_\beta = (1+\beta^2)\cdot \frac{\text{Precision} \cdot \text{Recall}}{\beta^2 \cdot \text{Precision}+\text{Recall}}
\end{equation*}

A specific case of this equation with $\beta=1$ is mathematically equivalent to the DICE Similarity Coefficient. A review found that DICE is the most popular evaluation metric amongst 2021 studies~\cite{review-metrics,evaluation-metrics, Sherer2021-le}.

% \begin{align*}
%   F_1 = \text{DICE} & = \frac{2 \cdot \text{Precision} \cdot \text{Recall}} {\text{Precision} + \text{Recall}} & \\ 
%   % & = \frac{2 \cdot \frac{TP}{TP + FP} \cdot \frac{TP}{TP + FN}}{\frac{TP}{TP + FP} + \frac{TP}{TP + FN}} \\
%   % & = \frac{2 \cdot TP \cdot TP}{TP(TP+FN)+TP(TP+FP)} \\ 
%   & = \frac{2 \cdot TP}{2TP + FP + FN} \quad = \quad \frac{2|S_g^1\cap S_p^1|}{|S_g^1|+|S_p^1|} , \text{where} \begin{cases}
%     S_g^1 & \text{ground truth} \\
%     S_p^1 & \text{segmentation}
%   \end{cases}
% \end{align*}

\begin{equation*}
 F_1 = \text{DICE} = \frac{2 \cdot \text{Precision} \cdot \text{Recall}} {\text{Precision} + \text{Recall}} = \frac{2 \cdot TP}{2TP + FP + FN} \quad = \quad \frac{2|S_g\cap S_p|}{|S_g|+|S_p|}
\end{equation*}

Where $S_g$ is the ground truth segmentation and $S_p$ is the predicted segmentation. From this relationship, the DICE score has found popularity in image segmentation for similar reasons that the $F_1$ score has found its popularity in classical machine learning; it can provide a fair result for imbalanced datasets. This mentality is applicable in our scenario because a tumour will make up very little of the total volume of the domain space. This argument extends to a Volumetric DSC by considering the above in all three dimensions~\cite{APL}.

Another popular related evaluation method is the Jaccard Index, which measures the intersection over the union of two sets:

\begin{equation*}
 \text{JAC} = \frac{TP}{TP+FP+FN} = \frac{|S_g\cap S_p|}{|S_g \cup S_p|} \iff \frac{DICE}{2 - DICE}
\end{equation*}

Since the numerator for the Jaccard Index is smaller than the DICE (since we avoid the issue of counting the intersecting sections twice), the JAC is always larger than the DICE score.

\subsection{Surface Based}\label{sect:surface-based}

Also commonly known as Boundary-Distance-Based Methods~\cite{boundary-overlap-metrics} compares the distance between two structure
surfaces. These can be maximum, average or distance at a set percentile of ordered distances~\cite{evaluation-metrics}.

A typical example is the Haussdorf Distance. Here, a directed distance metric is the maximum distance from a point in the first set to the nearest point in the other between two individual voxels~\cite{boundary-overlap-metrics}. Therefore, the better the HD metric, the smaller the value it returns. Here, the distance is typically Euclidian distance.

\begin{equation*}
 \text{HD}(A,B) = \max(h(A,B), h(B,A)), \quad \text{ and directed h}(A,B)=\max_{a\in A}\min_{b \in B} ||a-b||
\end{equation*}

The HD is generally sensitive to outliers; therefore, direct HD application gives uninspiring results because noise and outliers are common in medical segmentations~\cite{boundary-overlap-metrics}. Therefore, we can calculate the average directed Haussdorf Distance.

\subsection{Volume Based}

Volume-based metrics consider only the volume of the segmentation~\cite{evaluation-of-metrics-in-prostate,review-metrics, boundary-overlap-metrics}. However, its poor spatial descriptions make it more commonly used jointly with other metrics.

\begin{equation*}
 \text{Relative Volume Difference (RVD)} = \bigg| \frac{|S_g|-|S_p|}{|S_g|}\bigg|
\end{equation*}

\subsection{Evaluation}\label{sect:evaluation-of-evaluation-methods}

All these methods can be advantageous in some places rather than others. To decide which segmentation is best, we can list some challenging scenarios.

\begin{figure}[H]
  \centering
  \includegraphics[width=\linewidth]{../figures/segmentation-cases-1.png}
  \caption{Figure from~\cite{boundary-overlap-metrics} illustrating cases of segmentation to aid with explanation of set-backs of certain evaluation metrics}\label{fig:segmentation-cases-1}
\end{figure}

\begin{itemize}
  \item Classification Based (Section~\ref{sect:classification-based}) and Spatial Overlap Based (Section~\ref{sect:spatial-overlap-based}) are similar; they are concerned with the number of correctly classified or misclassified voxels without taking into account their spatial distribution. Here, Figure~\ref{fig:segmentation-cases-1}(a) and Figure~\ref{fig:segmentation-cases-1}(c) would achieve similar results despite Figure~\ref{fig:segmentation-cases-1}(a) being locally bound to a better area.
  \item With Haussdorf Distance (Section~\ref{sect:surface-based}) output segmentations generated by Figure~\ref{fig:segmentation-cases-1}(d) and Figure~\ref{fig:segmentation-cases-1}(e) will result in the same score, which is not favourable in a radiotherapy planning environment where an organ-at-risk is involved.
  \item Figure~\ref{fig:segmentation-cases-1}(b) would score flawlessly when using volumetric score estimation. However, it does not consider spatial placement, making this measurement poor when used individually.
\end{itemize}

\subsection{Estimated Editing Based}\label{sect:surface-dice}

Selecting a measurement that can reflect a clinician's acceptability score is difficult. A study found a lack of correlation between a geometric index and expert evaluation, with the JAC score having a 13\% False Positive Rate. The study's conclusion summarised that scores such as JSC and volumetric DSC ``provide limited clinical context and correlation with clinical or dosimetric quality''~\cite{Sherer2021-le}.

% Because of the clinical context of evaluating the segmentation by a machine, it may sometimes be helpful to define a performance metric as the ``fraction of the surface that needs to be redrawn''~\cite{Nikolov2021-xe} since models at this point require manual review to avoid automation bias (Section~\ref{sect:using-the-tool}). This method is helpful for larger structures as it doesn't assign much weight to the large trivial internal volume, which accounts for a much more significant proportion of the score.

\subsubsection{Surface DSC}\label{sect:surface-DSC}

The study at~\cite{Sherer2021-le} helped drive an initiative to combine aspects of surface Based evaluation (Section~\ref{sect:surface-based}) and Spatial Overlap Based evaluation (Section~\ref{sect:spatial-overlap-based}) into a Surface DICE which assesses the specified tolerance instead of the overlap of the two volumes.

\begin{figure}
  \centering
  \includegraphics[width=0.3\linewidth]{../figures/Surface-dice.png}
  \caption{Taken from~\cite{Nikolov2021-xe}. Illustrates the computation of the surface DICE, where the continuous line is the predicted surface, and the dashed line is the ground truth. The black arrows show the maximum deviation tolerated without penalty; therefore, in pink are the unacceptable deviations and green otherwise.}\label{fig:surface-dice}
\end{figure}

We can formulate the Surface DSC score in a mathematical definition~\cite{Sherer2021-le} with its corresponding illustration in Figure~\ref{fig:surface-dice}.

\begin{equation*}
 \text{Surface DSC} = \frac{|S_p \cap B_{g,\tau}| + |S_g \cap B_{p,\tau}|}{|S_p| + |S_g|}
\end{equation*}

This definition measures the agreement between just the surfaces of two structures above a clinically determined tolerance parameter, $\tau$. Here, $B_{p,\tau}$ represents the boundary region of the predicted surface within a maximum margin of deviation $\tau$ and similarly for $B_{g,\tau}$ for the ground truth.

\subsubsection{Added Path Length}

Similarly, the APL score predicts ``the path length of a contour that has to be added''~\cite{APL}. APL achieved similarly by considering the number of added voxels required between the prediction and the gold standard with no regard to tolerance as a pose to Surface DSC (Section~\ref{sect:surface-DSC})

% \begin{warning}
%  For future reference, \textit{\href{https://stackoverflow.com/questions/73286639/how-to-calculate-added-path-length-apl-image-segmentation-metric}{stack overflow discussion}}

%  Implementation of surface DSC and APL: \textit{\href{https://github.com/pyplati/platipy/blob/master/platipy/imaging/label/comparison.py}{source code}}
% \end{warning}

\subsection{Summary}

This is why we settle at the Surface DSC (Section~\ref{sect:surface-dice}), which prioritizes deviation along the boundary to a certain degree while measuring the fraction of the surface that needs to be redrawn, thus favouring a more conservative prediction of Figure~\ref{fig:segmentation-cases-1}(d) instead of (e).

For this project, we shall select an evaluation measurement more biased towards conservative boundary estimates not to touch the organs at risk. The clinician's review pipeline, in part, influenced this choice; it would be easier to correct Figure~\ref{fig:segmentation-cases-1}(d) instead of Figure~\ref{fig:segmentation-cases-1}(e) because correcting the latter would likely take a considerable amount of time as it would require redrawing almost all of the boundary, whereas the former could be corrected much faster~\cite{Nikolov2021-xe}.

%%%%%%%%%%%%%%%%%%%%%%%%%%%%%%%%%%%%
\chapter{Results and Discussion}\label{sect:results}\label{sect:discussion}

\section{Baseline -- nnUNet}

% The results of the baseline for each anatomy will be displayed alongside the result of the transfer to make it easier to reason about the effectiveness of the transfer technique. 



\section{Many-shot Transfer -- TotalSegmentator}

% total segmentator had 500 females to 700 males which may explain the struggle to locate the uterus and vagina.

\section{Few-shot Transfer -- UniverSeg}

\section{Zero-shot Transfer -- MedSAM}

\subsection{Training Bottlenecks}

\begin{figure}[H]
  \centering
  \subfigure[Raw epoch time before optimizations]{\includegraphics[width=.45\linewidth, trim=0 0 0 360px, clip]{../../data/MedSAM_finetuned/boxed_lowres/Anorectum/progress.png}}
  \subfigure[Epoch time after pipeline optimizations]{\includegraphics[width=.45\linewidth, trim=0 0 0 360px, clip]{../../data/MedSAM_finetuned/boxed_lowres_2/Anorectum/progress.png}}
  \caption{Comparison of epoch times for an expensive MedSAM model before and after pipeline optimizations.}\label{fig:training-comparison-med}
\end{figure}

This model requires three channels of $1024 \times 1024$ image slices, making each input image in its raw state a costly sample to send over the network especially after considering that each channel is a duplicate to match the requirements of the SAM model. This makes training on commodity hardware a difficult thing to accomplish. Training times can be very long without adding changes to the training pipeline. Figure~\ref{fig:training-comparison-med} compares training pipelines and performance improvements. 

Firstly, image slices were stored in a $(1 \times 512 \times 512)$ resolution and upsampled before the forward pass. Furthermore, caching~\cite{charlie-caching} stored these smaller images on the training node to avoid repeated and expensive reads over the network. This improved training epoch times by 50 times.

\subsection{Point Based Transfer}

\subsection{Box Based Transfer}


% Mention SAM's argument about ignoring small structures

%%%%%%%%%%%%%%%%%%%%%%%%%%%%%%%%%%%%
\chapter{Conclusion}

Transfer works!

% because of the way the CTV is drawn based on the progression of the tumor and how far along the patient is, this influences the contour drawen. Perhaps, the metrics suggest that a model with only one modality is not enough to make a decision which sparks the idea of a model which is capable of multi-modal analysis.

%%%%%%%%%%%%%%%%%%%%%%%%%%%%%%%%%%%%
\chapter{Ethics}

The lack of effort to protect the identities and confidentiality of patients during research projects may result in ``stigma, embarrassment, and discrimination''~\cite{health-privacy} if the data is misused.
This project involves the intimate and personal information of many female patients whose privacy must be protected before research occurs.

\section{Patient disclosures}

Researchers may collaborate with third parties, such as Imperial College London, by providing anonymised data that third parties cannot reverse-engineer to identify the patient. The collaborating hospital, The Royal Marsden Hospital, does not require ``explicit consent'' for sharing collected clinical data with outside entities as long as the patient is made aware of the ways their ``de-identified/anonymised'' data may be used.~\cite{royal-marsden-privacy-note}. Imperial College's Medical Imaging team also arranges formalities, such as acting as "ethical data stewards"~\cite{Larson2020-ib}. 

The MIRA team acts as responsible data stewards by storing anonymised data within a folder on the college network. They received all provided data in the \texttt{NIfTI} file format, which discloses no personally identifiable information, as defined by the GOV website~\cite{gov-gdpr}. Specific access rights limit data availability in this folder, ensuring security measures. Moreover, taking the data outside this folder reduces individual patient risk through the exchange of de-identified data.

Without such disclosure, anonymisation, and a security guarnatee of the data, patients may be reluctant to provide candid and complete disclosures of their sensitive information, even to physicians, which may prevent a complete diagnosis if their data is not maintained anonymously.

\section{Using the tool}\label{sect:using-the-tool}

The applications of this tool bode well in the healthcare ecosystem as the community slowly accepts the involvement of AI-powered medical tools. Radiology is one application that has been most welcoming of the new technological advances as there is potential for substantial aid by reducing manual labour, increasing precision and freeing up the primary care physician's time~\cite{Amisha2019-ki}.

However, it is too early to take the results of the medical tool as gospel. For current cervical radiotherapy delineation tools, only 90\% of the output is acceptable for clinical use~\cite{LIU2020172}. Therefore, The remainder can potentially cause more harm than good if not checked properly. For example, the overlap of a PTV with an organ-at-risk may invoke a cascade of adverse effects for the patient. The remaining 10\% of outputs may score incorrectly because the model uses a single modality, but physicians may base their final judgement on a multivariate analysis. Therefore, clinicians should use the tool as a second opinion rather than a primary source of information. Otherwise, an ethical dilemma of establishing the responsible party for incorrect decisions made by DL tools should also be determined~\cite{Chen2021-dg}.

Clinicians can fall into the trap of automation bias as AI becomes more commonplace in clinical environments~\cite{STRAW2020101965}. However, many models of this age codify the existing bias in common cases, which often will fail those patients who do not fit the majority's expectations. 

Therefore, before integrating tools into workflows, a committee must establish the degree of supervision required from physicians if this tool is to be used in practice. Currently, oncologists will be required to reverse-engineer the `black box' results to verify why a decision has been made. 

%%%%%%%%%%%%%%%%%%%%%%%%%%%%%%%%%%%%

\clearpage

\appendix

\chapter{Appendix}

\section{Ground Truth Organ Delineations}

\subsection{Organs At Risk}

\begin{figure}[H]
  \centering
  \subfigure[Axial]{
    \includegraphics[width=0.3\textwidth]{../figures/PatientStructureExamples/Anorectum_003/Axial.png}\label{fig:example-anorectum-axial}
  }
  \subfigure[Coronal]{
    \includegraphics[width=0.3\textwidth]{../figures/PatientStructureExamples/Anorectum_003/Coronal.png}\label{fig:example-anorectum-coronal}
  }
  \subfigure[Sagittal]{
    \includegraphics[width=0.3\textwidth]{../figures/PatientStructureExamples/Anorectum_003/Sagittal.png}\label{fig:example-anorectum-sagittal}
  }
  \subfigure[Axial]{
    \includegraphics[width=0.3\textwidth]{../figures/PatientStructureExamples/Bladder_088/Axial.png}\label{fig:example-bladder-axial}
  }
  \subfigure[Coronal]{
    \includegraphics[width=0.3\textwidth]{../figures/PatientStructureExamples/Bladder_088/Coronal.png}\label{fig:example-bladder-coronal}
  }
  \subfigure[Sagittal]{
    \includegraphics[width=0.3\textwidth]{../figures/PatientStructureExamples/Bladder_088/Sagittal.png}\label{fig:example-bladder-sagittal}
  }
  \caption{Views of the segmentation (in red) of the Anorectum (\ref{fig:example-anorectum-axial}-\ref{fig:example-anorectum-sagittal}) and  the segmentation (in red) of the Bladder (\ref{fig:example-bladder-axial}-\ref{fig:example-bladder-sagittal}) of an arbitrary patient}
\end{figure}

\subsection{CTV volumes}

\begin{figure}[H]
  \centering
  \subfigure[Axial]{
    \includegraphics[width=0.3\textwidth]{../figures/PatientStructureExamples/CTVp_096/Axial.png}\label{fig:example-CTVp-axial}
  }
  \subfigure[Coronal]{
    \includegraphics[width=0.3\textwidth]{../figures/PatientStructureExamples/CTVp_096/Coronal.png}\label{fig:example-CTVp-coronal}
  }
  \subfigure[Sagittal]{
    \includegraphics[width=0.3\textwidth]{../figures/PatientStructureExamples/CTVp_096/Sagittal.png}\label{fig:example-CTVp-sagittal}
  }
  \caption{Views of a segmented (in red) CTVp of an arbitrary patient}\label{fig:example-CTVp}
\end{figure}

\begin{figure}[H]
  \centering
  \subfigure[Axial]{
    \includegraphics[width=0.3\textwidth]{../figures/PatientStructureExamples/CTVn_007/Axial.png}\label{fig:example-CTVn-axial}
  }
  \subfigure[Coronal]{
    \includegraphics[width=0.3\textwidth]{../figures/PatientStructureExamples/CTVn_007/Coronal.png}\label{fig:example-CTVn-coronal}
  }
  \subfigure[Sagittal]{
    \includegraphics[width=0.3\textwidth]{../figures/PatientStructureExamples/CTVn_007/Sagittal.png}\label{fig:example-CTVn-sagittal}
  }
  \caption{Views of a segmented (in red) CTVn of an arbitrary patient}\label{fig:example-CTVn}
\end{figure}

\subsection{Parametrium, Uteurs, and Vagina}

\begin{figure}[H]
  \subfigure[Axial]{
    \includegraphics[width=0.3\textwidth]{../figures/PatientStructureExamples/Parametrium_088/Axial.png}\label{fig:example-Parametrium-axial}
  }
  \subfigure[Coronal]{
    \includegraphics[width=0.3\textwidth]{../figures/PatientStructureExamples/Parametrium_088/Coronal.png}\label{fig:example-Parametrium-coronal}
  }
  \subfigure[Sagittal]{
    \includegraphics[width=0.3\textwidth]{../figures/PatientStructureExamples/Parametrium_088/Sagittal.png}\label{fig:example-Parametrium-saggital}
  }
  \caption{Views of a segmented (in red) Parametrium of an arbitrary patient}\label{fig:example-Parametrium}
\end{figure}

\begin{figure}[H]
  \centering
  \subfigure[Axial]{
    \includegraphics[width=0.3\textwidth]{../figures/PatientStructureExamples/Uterus_012/Axial.png}\label{fig:example-uterus-axial}
  }
  \subfigure[Coronal]{
    \includegraphics[width=0.3\textwidth]{../figures/PatientStructureExamples/Uterus_012/Coronal.png}\label{fig:example-uterus-coronal}
  }
  \subfigure[Sagittal]{
    \includegraphics[width=0.3\textwidth]{../figures/PatientStructureExamples/Uterus_012/Sagittal.png}\label{fig:example-uterus-sagittal}
  }
  \subfigure[Axial]{
    \includegraphics[width=0.3\textwidth]{../figures/PatientStructureExamples/Vagina_69/Axial.png}\label{fig:example-vagina-axial}
  }
  \subfigure[Coronal]{
    \includegraphics[width=0.3\textwidth]{../figures/PatientStructureExamples/Vagina_69/Coronal.png}\label{fig:example-vagina-coronal}
  }
  \subfigure[Sagittal]{
    \includegraphics[width=0.3\textwidth]{../figures/PatientStructureExamples/Vagina_69/Sagittal.png}\label{fig:example-vagina-sagittal}
  }
  \caption{Views of the segmentation (in red) of the Uterus (\ref{fig:example-uterus-axial}-\ref{fig:example-uterus-sagittal}) and  the segmentation (in red) of the Vagina (\ref{fig:example-vagina-axial}-\ref{fig:example-vagina-sagittal}) of an arbitrary patient}
\end{figure}

%%%%%%%%%%%%%%%%%%%%%%%%%%%%%%%%%%%%

%% bibliography
\bibliography{../.latex-templates/references}

\end{document}