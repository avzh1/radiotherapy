\documentclass[11pt]{article}
\usepackage[margin=1in]{geometry}
\usepackage{listings}
\usepackage{graphicx}
\usepackage{subfigure}
\usepackage{subcaption} % For subfigures
\usepackage{float} % for H option in figures

\title{Description of Source Data}
\author{Anton Zhitomirsky}

\setlength{\parindent}{0pt}

\usepackage{biblatex} %Imports biblatex package
\addbibresource{../../source/bibliography.bib} %Import the bibliography file

\begin{document}

\maketitle

\section{Structure of source Data}

Results are structured in the file:

\begin{lstlisting}[language=bash]
/vol/biomedic3/bglocker/nnUNet
\end{lstlisting}

\begin{lstlisting}[language=inform]
-rwxr-xr-x   1 bglocker biomedia 236 Sep 24 15:16 exports
drwxr-sr-x   9 bglocker biomedia   9 Nov 25 10:55 nnUNet_preprocessed
drwxr-sr-x   9 bglocker biomedia  10 Nov 25 10:50 nnUNet_raw
drwxr-sr-x   9 bglocker biomedia   9 Nov 25 12:20 nnUNet_results
drwxr-sr-x  11 bglocker biomedia  11 Dec 16 09:10 nnUNet_testing
-rw-r--r--   1 bglocker biomedia 644 Oct 20 07:20 run_nnunet_0.sh
-rw-r--r--   1 bglocker biomedia 644 Oct 20 07:20 run_nnunet_1.sh
-rw-r--r--   1 bglocker biomedia 644 Oct 20 07:20 run_nnunet_2.sh
-rw-r--r--   1 bglocker biomedia 644 Oct 20 07:21 run_nnunet_3.sh
-rw-r--r--   1 bglocker biomedia 644 Oct 20 07:21 run_nnunet_4.sh
\end{lstlisting}

\section*{nnUNet\_raw}

nnUNet\_raw has the original (training) images with manual annotations. Each Dataset below is treated as a binary segmentation problem. See Section\ref{section:itksnap}

\begin{lstlisting}[language=inform]
drwxr-sr-x  4 bglocker biomedia   5 Sep 17 13:47 Dataset001_Anorectum
drwxr-sr-x  3 bglocker biomedia   5 Sep 17 20:24 Dataset002_Bladder
drwxr-sr-x  3 bglocker biomedia   5 Sep 17 20:27 Dataset003_CTVn
drwxr-sr-x  3 bglocker biomedia   5 Sep 17 20:28 Dataset004_CTVp
drwxr-sr-x  3 bglocker biomedia   5 Sep 17 20:29 Dataset005_Parametrium
-rw-r--r--  1 bglocker biomedia 135 Nov 25 10:50 note
\end{lstlisting}

\subsection*{What is a Binary Segmentation Problem?}

\section{Viewing the Data}\label{section:itksnap}

\subsection*{ItkSnap}

The viewing tool used is ItkSnap, which was developed as an open source tool for viewing medical imaging scans. The view (Figure\ref{fig:view}) shows how you would see input data.

\begin{figure}[H]
    \centering
    \includegraphics*[]{images/view.png}
    \caption{view of all input data}\label{fig:view}
\end{figure}

With that we can use this tool to view input data. Here, the R and L stand for right and left respectively, and the A and P stand for Anterior and Posterior. We can provide a few other examples of viewing data displayed below in Figure\ref{fig:AnorectumImage}. We are further provided with manual annotation of the substructures. Figure\ref{fig:AnorectumLabel} shows an example of the annotation of the Anorectum. You can enable the 3D visual model through \texttt{Edit > 3D Panel > Toggle 3D view}.

\begin{figure}[H]
    \centering
    \captionsetup{width=0.45\textwidth}
    \begin{minipage}{.5\textwidth}
        \centering
        \includegraphics[width=\linewidth]{images/AnorectumImage.png}
        \caption{ItkSnap view of the Anorectum Raw Image}\label{fig:AnorectumImage}
    \end{minipage}%
    \begin{minipage}{.5\textwidth}
        \centering
        \includegraphics[width=\linewidth]{images/AnorectumLabel.png}
        \caption{ItkSnap view of the Anorectum Raw Image}\label{fig:AnorectumLabel}
    \end{minipage}%
\end{figure} 

\end{document}